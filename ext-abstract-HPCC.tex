\documentclass[jocse]{jocseart}

\usepackage{booktabs} % For formal tables


% Copyright
\setcopyright{jocsecopyright}

% DOI
\jocseDOI{10.22369/issn.2153-4136/x/x/x }

% the following commands remove page numbers and headers that JOCSE formatting adds 
% in preparation for publication DO NOT REMOVE THESE
\pagestyle{plain} 
\pagenumbering{gobble}

\begin{document}
% Title info
\title{HPC Carpentry: Recent Progress and Incubation Toward an Official Carpentries Lesson Program}
%\titlenote{This format mimics the ACM SIG Proceedings format.}

% Author(s) info

\author{Andrew Reid}
%\authornote{First Author insisted his name be first.}
%\orcid{0000-}
\affiliation{%
  \institution{US National Institute of Standards and Technology}
  \streetaddress{}
  \city{Gaithersburg}
  \state{Maryland}
  \postcode{20899}
}
\email{andrew.reid@nist.gov}


\author{Trevor Keller}
%\authornote{First Author insisted his name be first.}
%\orcid{0000-}
\affiliation{%
  \institution{US National Institute of Standards and Technology}
  \streetaddress{}
  \city{Gaithersburg}
  \state{Maryland}
  \postcode{20899}
}
\email{trevor.keller@nist.gov}


\author{Alan O'Cais}
%\authornote{First Author insisted his name be first.}
%\orcid{0000-}
\affiliation{%
  \institution{University of Barcelona}
  \streetaddress{}
  \city{Barcelona}
  \state{Spain}
  \postcode{08007}
}
\email{alan.ocais@gmail.com}


\author{Annajiat Alim Rasel}
%\authornote{First Author insisted his name be first.}
%\orcid{0000-}
\affiliation{%
  \institution{BRAC University}
  \streetaddress{}
  \city{Dhaka}
  \state{Bangladesh}
  \postcode{1212}
}
\email{annajiat@bracu.ac.bd}


\author{Wirawan Purwanto}
%\authornote{First Author insisted his name be first.}
\orcid{0000-0002-2124-4552}
\affiliation{%
  \institution{Old Dominion University}
  \streetaddress{5115 Hampton Blvd}
  \city{Norfolk}
  \state{Virginia}
  \postcode{23529}
}
\email{wpurwant@odu.edu}

\author{Jane Herriman}
\orcid{0000-0003-4769-1403}
\affiliation{%
  \institution{Lawrence Livermore National Laboratory}
  \streetaddress{}
  \city{Livermore}
  \state{California}
  \postcode{94550}
}
\email{herriman1@llnl.gov}

\author{Benson Muite}
%\authornote{First Author insisted his name be first.}
%\orcid{}
\affiliation{%
  \institution{Kichakato Kizito}
  \streetaddress{}
  \city{Nairobi}
  \state{Kenya}
  \postcode{}
}
\email{benson_muite@emailplus.org}



% The default list of authors is too long for headers.
\renewcommand{\shortauthors}{Reid et al.}

% Abstract 
\begin{abstract}
The HPC Carpentry project aims to develop highly interactive workshop training materials to empower novice users of HPC resources to effectively leverage HPC to solve scientific and technical problems of interest to them. Modeled after the Carpentries' training programs, the workshop setting provides learners with a hands-on experience that elicits confidence in their ability to work with the system and sufficient vocabulary and review materials to make subsequent self-study more effective.

The project's most recent focus has been on developing workshop materials that take learners from the command-line shell (using the existing Software Carpentry's Unix Shell lesson), followed by our Introduction to HPC lesson, covering remote access and resource management, and then to a lesson on HPC workflow management, which walks learners through the specification and execution of a scaling study on an HPC system, emphasizing both the benefits and limitations of HPC systems for domain applications. This workshop program was recently run in full at the Lawrence Livermore National Laboratory.

The project plans to develop additional training resources for HPC developers, which will give instructors the option of replacing the workflow lesson with lessons on parallel frameworks, such as MPI.

In a major milestone, the current steering committee is leading the project through the formal incubation process towards becoming an official Carpentries lesson program alongside the existing Software, Data, and Library Carpentry programs.
\end{abstract}


% Keywords
\keywords{Cyberinfrastructure, training, pedagogy, HPC, parallel computing, big data, non-degree training, hands-on}

% Generate the title
\maketitle

% Input the body of the paper by providing the file name. For example,
% if the file name is body.tex, then input{body}
\section{Background}

HPC Carpentry~\cite{HPCC-website} is an informal training project with a mission to provide a set of lessons aimed at introducing the basic ``know-how'' of running applications on high-performance computing (HPC) resources to new audiences, including investigators from fields which are not traditional users of HPC systems, as well as novice users from fields in which HPC is commonly used.
Eventually, the project's goal is to empower HPC novices to effectively leverage HPC to solve scientific and technical problems in their respective domains.
The project paves the way for the potential users from non-traditional HPC disciplines to tap into HPC resources for their data analysis, modeling, and simulation needs while remaining relevant for beginners from the traditional HPC disciplines.
The current project is the product of significant work over the past several years, incorporating valuable materials from many contributors.



\section{Lesson Development Efforts}

The recent focus of HPC Carpentry has been the development a complete workshop program for new HPC users.
We begin with an introduction to the command-line shell using Software Carpentry's Unix Shell lesson~\cite{SwC-UNIX-Shell_v3},
followed by our Introduction to High Performance Computing lesson~\cite{HPCC_hpc-intro}, covering remote access and resource management.
We end with a newly developed lesson on HPC workflow management~\cite{HPCC_hpc-workflows}, which walks learners through the specification and execution of a scaling study on an HPC system, emphasizing both the benefits and limitations of HPC systems for domain applications. 
This set of three lessons is sufficient to offer a two-day hands-on workshop in a format similar to that of the Software Carpentry or Data Carpentry workshops.

The project plans to develop more advanced training resources for HPC developers, which will give workshop instructors the option to substitute the workflow lesson with a coding exercise in a parallel framework (such as MPI), for example.
Several lessons had also been developed in the past, including an introduction of parallel programming using the Chapel programming language.
Furthermore, we have received engagement from HPC community members at large who explore the potential of merging their in-house lessons into HPC Carpentry's lesson portfolio.
This is still an ongoing effort and engagement with the community.


\section{Recent Workshops}

The complete HPC Carpentry workshop program for new users was recently offered at the Lawrence Livermore National Laboratory in June of 2024.
Previous workshops were held at University College Dublin, Brac University, Helmholtz Einstein International Berlin Research School in Data Science (HEIBRiDS), University of Mauritius, Florida International University, Delft University of Technology, National Institute of Standards and Technology, and EPFL CECAM.

Feedback from these workshops has been crucial in improving the lesson material. From the run-up to the EPFL CECAM workshop, we discovered some important version-specific issues in the workflow tool we chose to use, which required modifications to the draft lesson to accommodate. 

From the Lawrence Livermore workshop, we heard from learners that the process of buliding up a workflow configuration file is sensitive to a loss of context --- if a learner misses a step, it's hard to recover, because the next version of the file d 
% \textbf{FIXME: Can we say briefly what the same things about these workshops and what we learned after offering these workshops?}

One of the important issue we are facing in offering HPC Carpentry workshops is the need for HPC infrastructure for learners to use during the workshop.
While some HPC site operators have their own HPC systems to conduct their own workshop, others, particularly from under-resourced institutions, do not have their own HPC resources.
We have attempted to create a small HPC Carpentry cluster in the cloud.
More recently, we have acquired support from Jetstream 2 through ACCESS to set up a ``standard'' reference HPC Carpentry cluster in a virtual-machine-based environment.
The cluster set-up has been prepared in an automated fashion using Terraform scripting.
This effort could pave the way to allow instructors to set their own clusters, irrespective of the existence of a local HPC cluster in their own institution.


\section{Incubation to the Carpentries Lesson Program}

In a major milestone, the steering committee is leading the project through the formal incubation process towards becoming an official Carpentries lesson program alongside the existing Software, Data, and Library Carpentry programs.
This process is expected to last for about 18 months, with an expected target date of December 2025.


\section{Acknowledgment}

We acknowledge the infrastructure support hosted on Jetstream 2 cloud environment. Allocation to Jetstream 2 was provided through ACCESS, which is funded by the US National Science Foundation.


\bibliographystyle{ACM-Reference-Format}
\bibliography{deapsecure}

\end{document}
